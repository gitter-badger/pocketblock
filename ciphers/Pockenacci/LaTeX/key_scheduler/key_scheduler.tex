%% PocketBlock | Fibonacci Key Scheduler Demo for Pockenacci
%% (C) 2017 Justin Troutman | GPLv3

\documentclass[border=1cm]{standalone}

\usepackage{xcolor}

\usepackage{tikz}
\usetikzlibrary{positioning}

%% set global font

\usepackage{PTSansCaption}
\renewcommand*\familydefault{\sfdefault} %% Only if the base font of the document is to be sans serif
\usepackage[T1]{fontenc}

%% PocketBlock | mgzy colorway | (C) 2016 Justin Troutman | GPLv3

%% Do not build this file; it's to be loaded via ``\input{}'' in a parent file.
%% Before your \begin{document} declaration, type: %% PocketBlock | mgzy colorway | (C) 2016 Justin Troutman | GPLv3

%% Do not build this file; it's to be loaded via ``\input{}'' in a parent file.
%% Before your \begin{document} declaration, type: %% PocketBlock | mgzy colorway | (C) 2016 Justin Troutman | GPLv3

%% Do not build this file; it's to be loaded via ``\input{}'' in a parent file.
%% Before your \begin{document} declaration, type: \input{colorway.tex}

%% Set font, along with draw line widths and joints for entire 6x6 grid

\tikzstyle{pockenacci}=[on grid,
                  font=\bfseries,
                  line width=.6mm,
                  line join=bevel]

%% Set draw lines to white, with a fill opacity of 65

\tikzstyle{rest}=[white,
                  fill=rest!65]

\tikzstyle{play}=[white,
                  fill=play!65]

%% Define RGB values for play and rest fills

\definecolor{play}{RGB}{47,16,105} % purple
\definecolor{rest}{RGB}{85,168,176} % teal


%% Set font, along with draw line widths and joints for entire 6x6 grid

\tikzstyle{pockenacci}=[on grid,
                  font=\bfseries,
                  line width=.6mm,
                  line join=bevel]

%% Set draw lines to white, with a fill opacity of 65

\tikzstyle{rest}=[white,
                  fill=rest!65]

\tikzstyle{play}=[white,
                  fill=play!65]

%% Define RGB values for play and rest fills

\definecolor{play}{RGB}{47,16,105} % purple
\definecolor{rest}{RGB}{85,168,176} % teal


%% Set font, along with draw line widths and joints for entire 6x6 grid

\tikzstyle{pockenacci}=[on grid,
                  font=\bfseries,
                  line width=.6mm,
                  line join=bevel]

%% Set draw lines to white, with a fill opacity of 65

\tikzstyle{rest}=[white,
                  fill=rest!65]

\tikzstyle{play}=[white,
                  fill=play!65]

%% Define RGB values for play and rest fills

\definecolor{play}{RGB}{47,16,105} % purple
\definecolor{rest}{RGB}{85,168,176} % teal
 % user-customizable colorway scheme; default is mgzy @ 65 opacity

\begin{document}

%% define command for generating six random initial seed values

\newcommand\getRandom{
  \pgfmathsetmacro{\ione}{random(0,9)}
  \pgfmathsetmacro{\itwo}{random(0,9)}
  \pgfmathsetmacro{\ithree}{random(0,9)}
  \pgfmathsetmacro{\ifour}{random(0,9)}
  \pgfmathsetmacro{\ifive}{random(0,9)}
  \pgfmathsetmacro{\isix}{random(0,9)}
}

%% perform calculations for initial seed; 6 random numbers between 0-9

\getRandom %% pull random data for use in key scheduling
 % generates six random numbers to seed key scheduler; default; comment out to turn off

%% %% define command for manually inputting six user-defined initial seed values; numbers between 0-9

\newcommand{\ione}{1}
\newcommand{\itwo}{2}
\newcommand{\ithree}{3}
\newcommand{\ifour}{4}
\newcommand{\ifive}{5}
\newcommand{\isix}{6}
 % user defines six desired numbers to seed key scheduler; uncomment to turn on

\pagecolor{white} %% set to match white draw lines for blocks in colorway.tex

%% set up 6x6 grid style (colorway.tex) for Pockenacci block cipher

\begin{tikzpicture}[pockenacci]

%% draw 2D blocks with white borders and custom fill (colorway.tex) at given coordinates

%% first row (top)
\draw[rest] (0,5) rectangle (1,6); %% row 1, column 1 (leftmost; top left corner)
\draw[rest] (1,5) rectangle (2,6); %% row 1, column 2
\draw[rest] (2,5) rectangle (3,6); %% row 1, column 3
\draw[rest] (3,5) rectangle (4,6); %% row 1, column 4
\draw[rest] (4,5) rectangle (5,6); %% row 1, column 5
\draw[rest] (5,5) rectangle (6,6); %% row 1, column 6 (rightmost; top right corner)

%% second row
\draw[rest] (0,4) rectangle (1,5); %% row 2, column 1
\draw[rest] (1,4) rectangle (2,5); %% row 2, column 2
\draw[rest] (2,4) rectangle (3,5); %% row 2, column 3
\draw[rest] (3,4) rectangle (4,5); %% row 2, column 4
\draw[rest] (4,4) rectangle (5,5); %% row 2, column 5
\draw[rest] (5,4) rectangle (6,5); %% row 2, column 6

%% third row
\draw[rest] (0,3) rectangle (1,4); %% row 3, column 1
\draw[rest] (1,3) rectangle (2,4); %% row 3, column 2
\draw[rest] (2,3) rectangle (3,4); %% row 3, column 3
\draw[rest] (3,3) rectangle (4,4); %% row 3, column 4
\draw[rest] (4,3) rectangle (5,4); %% row 3, column 5
\draw[rest] (5,3) rectangle (6,4); %% row 3, column 6

%% fourth row
\draw[rest] (0,2) rectangle (1,3); %% row 4, column 1
\draw[rest] (1,2) rectangle (2,3); %% row 4, column 2
\draw[rest] (2,2) rectangle (3,3); %% row 4, column 3
\draw[rest] (3,2) rectangle (4,3); %% row 4, column 4
\draw[rest] (4,2) rectangle (5,3); %% row 4, column 5
\draw[rest] (5,2) rectangle (6,3); %% row 4, column 6

%% fifth row
\draw[rest] (0,1) rectangle (1,2); %% row 5, column 1
\draw[rest] (1,1) rectangle (2,2); %% row 5, column 2
\draw[rest] (2,1) rectangle (3,2); %% row 5, column 3
\draw[rest] (3,1) rectangle (4,2); %% row 5, column 4
\draw[rest] (4,1) rectangle (5,2); %% row 5, column 5
\draw[rest] (5,1) rectangle (6,2); %% row 5, column 6

%% sixth row (bottom)
\draw[rest] (0,0) rectangle (1,1); %% row 6, column 1 (leftmost; bottom left corner)
\draw[rest] (1,1) rectangle (2,0); %% row 6, column 2
\draw[rest] (2,1) rectangle (3,0); %% row 6, column 3
\draw[rest] (3,1) rectangle (4,0); %% row 6, column 4
\draw[rest] (4,1) rectangle (5,0); %% row 6, column 5
\draw[rest] (5,1) rectangle (6,0); %% row 6, column 6 (rightmost; bottom right corner)

%% perform calculations for all subkeys; 36 total

%% "intuitive version" for LaTeX/TikZ novices

\pgfmathtruncatemacro{\kone}{mod(\ione + \itwo,10)}
\pgfmathtruncatemacro{\ktwo}{mod(\itwo + \ithree,10)}
\pgfmathtruncatemacro{\kthree}{mod(\ithree + \ifour,10)}
\pgfmathtruncatemacro{\kfour}{mod(\ifour + \ifive,10)}
\pgfmathtruncatemacro{\kfive}{mod(\ifive + \isix,10)}
\pgfmathtruncatemacro{\ksix}{mod(\isix + \ione,10)}

\pgfmathtruncatemacro{\kseven}{mod(\kone + \ktwo,10)}
\pgfmathtruncatemacro{\keight}{mod(\ktwo + \kthree,10)}
\pgfmathtruncatemacro{\knine}{mod(\kthree + \kfour,10)}
\pgfmathtruncatemacro{\kten}{mod(\kfour + \kfive,10)}
\pgfmathtruncatemacro{\keleven}{mod(\kfive + \ksix,10)}
\pgfmathtruncatemacro{\ktwelve}{mod(\ksix + \kone,10)}

\pgfmathtruncatemacro{\kthirteen}{mod(\kseven + \keight,10)}
\pgfmathtruncatemacro{\kfourteen}{mod(\keight + \knine,10)}
\pgfmathtruncatemacro{\kfifteen}{mod(\knine + \kten,10)}
\pgfmathtruncatemacro{\ksixteen}{mod(\kten + \keleven,10)}
\pgfmathtruncatemacro{\kseventeen}{mod(\keleven + \ktwelve,10)}
\pgfmathtruncatemacro{\keighteen}{mod(\ktwelve + \kseven,10)}

\pgfmathtruncatemacro{\knineteen}{mod(\kthirteen + \kfourteen,10)}
\pgfmathtruncatemacro{\ktwenty}{mod(\kfourteen + \kfifteen,10)}
\pgfmathtruncatemacro{\ktwentyone}{mod(\kfifteen + \ksixteen,10)}
\pgfmathtruncatemacro{\ktwentytwo}{mod(\ksixteen + \kseventeen,10)}
\pgfmathtruncatemacro{\ktwentythree}{mod(\kseventeen + \keighteen,10)}
\pgfmathtruncatemacro{\ktwentyfour}{mod(\keighteen + \kthirteen,10)}

\pgfmathtruncatemacro{\ktwentyfive}{mod(\knineteen + \ktwenty,10)}
\pgfmathtruncatemacro{\ktwentysix}{mod(\ktwenty + \ktwentyone,10)}
\pgfmathtruncatemacro{\ktwentyseven}{mod(\ktwentyone + \ktwentytwo,10)}
\pgfmathtruncatemacro{\ktwentyeight}{mod(\ktwentytwo + \ktwentythree,10)}
\pgfmathtruncatemacro{\ktwentynine}{mod(\ktwentythree + \ktwentyfour,10)}
\pgfmathtruncatemacro{\kthirty}{mod(\ktwentyfour + \knineteen,10)}

\pgfmathtruncatemacro{\kthirtyone}{mod(\ktwentyfive + \ktwentysix,10)}
\pgfmathtruncatemacro{\kthirtytwo}{mod(\ktwentysix + \ktwentyseven,10)}
\pgfmathtruncatemacro{\kthirtythree}{mod(\ktwentyseven + \ktwentyeight,10)}
\pgfmathtruncatemacro{\kthirtyfour}{mod(\ktwentyeight + \ktwentynine,10)}
\pgfmathtruncatemacro{\kthirtyfive}{mod(\ktwentynine + \kthirty,10)}
\pgfmathtruncatemacro{\kthirtysix}{mod(\kthirty + \ktwentyfive,10)}

%% place calculated node values within node blocks

\node[nodetext] at (0.5,5.5) {\kone};
\node[nodetext] at (1.5,5.5) {\ktwo};
\node[nodetext] at (2.5,5.5) {\kthree};
\node[nodetext] at (3.5,5.5) {\kfour};
\node[nodetext] at (4.5,5.5) {\kfive};
\node[nodetext] at (5.5,5.5) {\ksix};

\node[nodetext] at (0.5,4.5) {\kseven};
\node[nodetext] at (1.5,4.5) {\keight};
\node[nodetext] at (2.5,4.5) {\knine};
\node[nodetext] at (3.5,4.5) {\kten};
\node[nodetext] at (4.5,4.5) {\keleven};
\node[nodetext] at (5.5,4.5) {\ktwelve};

\node[nodetext] at (0.5,3.5) {\kthirteen};
\node[nodetext] at (1.5,3.5) {\kfourteen};
\node[nodetext] at (2.5,3.5) {\kfifteen};
\node[nodetext] at (3.5,3.5) {\ksixteen};
\node[nodetext] at (4.5,3.5) {\kseventeen};
\node[nodetext] at (5.5,3.5) {\keighteen};

\node[nodetext] at (0.5,2.5) {\knineteen};
\node[nodetext] at (1.5,2.5) {\ktwenty};
\node[nodetext] at (2.5,2.5) {\ktwentyone};
\node[nodetext] at (3.5,2.5) {\ktwentytwo};
\node[nodetext] at (4.5,2.5) {\ktwentythree};
\node[nodetext] at (5.5,2.5) {\ktwentyfour};

\node[nodetext] at (0.5,1.5) {\ktwentyfive};
\node[nodetext] at (1.5,1.5) {\ktwentysix};
\node[nodetext] at (2.5,1.5) {\ktwentyseven};
\node[nodetext] at (3.5,1.5) {\ktwentyeight};
\node[nodetext] at (4.5,1.5) {\ktwentynine};
\node[nodetext] at (5.5,1.5) {\kthirty};

\node[nodetext] at (0.5,0.5) {\kthirtyone};
\node[nodetext] at (1.5,0.5) {\kthirtytwo};
\node[nodetext] at (2.5,0.5) {\kthirtythree};
\node[nodetext] at (3.5,0.5) {\kthirtyfour};
\node[nodetext] at (4.5,0.5) {\kthirtyfive};
\node[nodetext] at (5.5,0.5) {\kthirtysix};


%% place initial values at y coordinate positions

\node[coord] at (0.5,6.7) {\ione};
\node[coord] at (1.5,6.7) {\itwo};
\node[coord] at (2.5,6.7) {\ithree};
\node[coord] at (3.5,6.7) {\ifour};
\node[coord] at (4.5,6.7) {\ifive};
\node[coord] at (5.5,6.7) {\isix};


\end{tikzpicture}

\end{document}
